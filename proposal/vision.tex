\section{Vision}\label{sec:vison}
There is not a single writer who hasn’t dealt with the issue of writer's block.
It’s a problem that can destroy someone's creative potential for extended periods
of time with seemingly little recourse.  Writers often try to overcome this hurdle
and to generate inspiration for their stories with collaborative writing games,
which require the participation of multiple people in order to help break down
the participants’ creative barriers. Our project seeks to provide an alternative
to these cooperative writing games by providing solo writers with an artificial
writing partner.

We propose a neural network that will generate inspiration during the creative
writing process to help mitigate the effects of or even prevent writer's block
\cite{clark}. This neural network, under the working title of WriterBot, will
function as an in-the-loop aid to the writer in an interactive system where both
parties take turns writing sentences. The writer will write a sentence and
WriterBot will generate another sentence based on the writer’s input to serve
as the next line of the story. The writer will then have a chance to edit or
delete WriterBot’s sentence before continuing onto their own next sentence,
ultimately creating a story that is the product of both machine and human
inspiration.

WriterBot will be trained on a library of different fantasy authors. It will be
constructed as a RNN LM (Recurrent Neural Network Language Model) that will take
input from both the sentence that the dataset has input (also known as the context)
and the words that it has generated. \cite{RNN_language_model}. Once the sentence
has been processed and the output generated, the weights attributed to each RNN
cell will be updated according to the loss and how accurately the cell predicted
the correct output.

When the user interacts with the trained RNN their sentence will function like
the dataset does, as the context, for each RNN cell and each cell will feed into
the other with the word it has generated. When this process is completed, the RNN
will have output a complete sentence that the user can then edit or delete at their
leisure. The user will then type another sentence and the RNN will take the new
sentence and all previous sentences as input and follow the same process.

We hope to extend Clark et al.’s\cite{clark} human-centric approach by giving users the
option to choose when to seek suggestions from the machine \cite{creative_help}
in contrast to suggestions being automatically generated by the machine after
each sentence input, thereby giving people the ability to control the level of
machine intrusiveness in their writing process.  We will also try to generalize
the original paper’s approach by exploiting information from a larger context
outside of just the user’s previous sentence, as well as by training the machine
with data from different genres of fiction.  Finally, we would like to extend the
original paper by providing a more robust suite of related tools, including prompt
generation and the ability to save and share WriterBot-generated stories.
We hope WriterBot will be able to inspire writers to take their stories in
new and creative directions, and to provide an enjoyable way to get unstuck
in the creative writing process.
