\section{Roadmap}\label{sec:roadmap}
We have two general phases of work, and we will divide them by term. This term is our neural network phase, and winter term will be our UI/UX and feature phase. 

By the end of this term, we want to have a working CLI for our RNN, as well as a
useable dataset that fits our needs. We plan on writing the majority of our code
in Python, so our first step will be to determine which libraries to use for numerical
computation and the neural net. As of October 16th, this step is complete.
Our larger step will be to decide and implement the inner structure of the RNN,
which we plan to have done by 8th week. We have a simpler testing model in use,
so current subgoals include ironing out IO (week 6) and potentially including the
Jetson \ref{sec:resources} for extra processing power (week 7).

Concurrently, we will get access to data. Unfortunately, we are unable to use the
fiction database that the original used\cite{clark}, so finding decent data and
scraping it is a priority, because training will require that data. Our current
solution is to use data from Project Gutenberg in our preliminary models as a stand-in
while better data is found. By 6th week we intend on having a functional scraper
that preprocesses text data for immediate neural net consumption.

Our winter term agenda is to implement a UI for our CLI. Since our code will be
in Python, our plan is to use Django to create a web app that can be hosted on
Heroku . We want this done in the first few weeks of winter term so we can work
on further feature development. Preliminary work is happening now on the website;
we plan on having a simple Django site up soon (6th week). This way, we can directly
approach the integration of the two project sections as soon as winter term begins.
These features are fleshed out more fully, as is our main goal in the \ref{sec:deliverables} section.

We view the RNN as the more technically difficult task in this project,
which is why we are placing very heavy priority on its completion as soon as
possible. While not trivial, a UI is something that all of us have some
experience with, and will require less technical time than perfecting our
language model, which, again, is the core of our project.
